\chapter{Background}
\chapternames{Tarek Higazi, Dukagjin Ramosaj}
\section{Blockchain}

A new concept for trustless, decentralized and transparent information processing and storage has been developed, introduced and already integrated into several mainstream applications in the last years. Bitcoin, developed by an unknown person or group under the name of Satoshi Nakamoto \cite{nakamoto2008bitcoin}, led this initiative back in 2008 with its open-source, peer-to-peer network which cryptographically stores records in a chain and serves as a distributed ledger, this was named the blockchain. The currency in which these operations were valued and paid for with is called Bitcoin, and it was the first decentralized digital currency.

Blockchains are essentially continuously growing lists of information which are linked together and secured using cryptography. They consist of specific properties such as immutability, transparency, and the facilitation of cryptographically-secured peer-recorded transactions between users, and which are ratified by consensus on the network.
The main benefit of blockchains is that they are designed to be incorruptible in the sense that the data they hold cannot be modified without mutual consent. The entire history of the blockchain is always stored and recorded on it and is verified by peers. It is possible for every user to parse the blockchain's history until its very beginning.

This technology has been so far mostly used in the financial sector in applications such as peer-to-peer systems, cryptography, consensus protocols, decentralized storage, decentralized processing and smart contracts. The consolidation of these concepts is what makes blockchains exceptionally innovative as a programmable platform and system at the same time.
Being able to build up trustless interactions, along with business disintermediation, continues to be one of the most important objectives of using blockchain technology.

\section{Ethereum and Smart Contracts}

Smart contracts are a feature which the Ethereum blockchain is based on. A smart contract is basically a computer program which can check for the fulfillment of certain preconfigured conditions and control the transfer of funds or other assets between several parties \cite{relatedWork38}. While a standard contract enforces its terms through its legality, a smart contract actualizes this enforcement automatically through network consensus and a cryptographically secured system. The contracts are stored on the blockchain and thereby serve as a decentralized middle man which stores records, enforces conditions and is completely neutral and transparent. The most popular platform which utilizes this technology is the Ethereum blockchain. It empowers developers to create their own smart contracts and their own decentralized blockchain applications \cite{relatedWork38}.

The runtime environment which smart contracts are processed is called the Ethereum Virtual Machine (EVM). It continuously runs on every node on the chain, which means that every task which is executed inside of the EVM is executed by each full node \cite{relatedWork38}. This is a crucial part of the Ethereum consensus model and has the advantage that any contract on the EVM can call other contract at zero cost, however, it also has its disadvantages that the computational steps on the EVM are exceptionally costly \cite{relatedWork38}. The EVM is also isolated in the sense that no outside framework or file system is accessible, this to ensure determinism. The smart contracts themselves are written using the Solidity programming language, which was designed for this purpose. Moreover, it allows users to create contracts which can be used for voting, crowdfunding, blind auctions, multi-signature wallets and more.

In order to ensure that the Ethereum network would not be abused or deliberately attacked, the Ethereum protocol charges a fee for every computational step. The way this cost is paid is through an attribute called ‘gas’. The fee or the price of the gas is determined simply via supply and demand; the users’ willingness to pay vs. the price for which the miners are willing to mine the next block in the blockchain. Basically, the way these fees work is that every transaction contains a 'gasPrice' attribute, which is the price per computational step, and the 'startGas' attribute, being the maximum amount of gas which the sender is willing to pay for the transaction. Therefore, at every execution of a transaction a prior evaluation of the transaction cost is done \cite{relatedWork38}.
$$ gasCost(Tx)= {gasPrice * startGas}$$
The usage of the EVM for the most part makes sense when it comes to running business logic applications ("if this, then that") such as confirming signatures or other cryptographic objects \cite{relatedWork39}. On the other hand, any utilization which incorporates using the EVM as a file storage platform or anything GUI related would not be practical due to the tremendous costs.

\section{Integrity Checks}
\sectionnames{Simon Fallnich}

	This section describes basic principles to verify the integrity of data from an untrusted source.

	% Hashing
	\subsection{Hashing}
	\label{subsec:hashing}

		A \emph{hash function} maps an input of arbitrary length to an ouput of fixed length, referred to as \emph{hash} \cite{menezes1996}.
		Although hash functions are used for a wide range of applications in the domain of computer science, the class of \emph{cryptographic hash functions} is primarily relevant for our application.
		A cryptographic hash function, or \emph{one-way function}, is a hash function which is infeasible to invert.
		This essential property lays the foundation for efficient integrity check mechanisms as the resulting hashes can be used to securely proof equality of the underlying data.

	% Merkle Tree
	\subsection{Merkle Tree}
	\label{subsec:merkle-tree}

		\begin{figure}
			\centering
				\begin{tikzpicture}[line cap=round, line join=round, yscale=0.8]
					% First layer
					\node [circle, draw=black, fill=white, minimum size = 1cm] (node-1-1) at (0, 0) {$H_0$};
					% Second layer
					\foreach \a [count=\x] in {1, 2}
						\node [circle, draw=black, fill=white, minimum size=1cm] (node-2-\x) at (-12 + 8 * \x, -2) {$H_\a$};
					% Third layer
					\foreach \a [count=\x] in {3, ..., 6}
					  \node [circle, draw=black, fill=white, minimum size=1cm] (node-3-\x) at (-10 + 4 * \x, -4) {$H_\a$};
					% Fourth layer
					\foreach \a [count=\x] in {7, 8}
					  \node [circle, draw=black, fill=white, minimum size=1cm] (node-4-\x) at (-9 + 2 * \x, -6) {$H_\a$};
					% First data layer
					\foreach \a [count=\x] in {0, ..., 2}
					  \node [draw=gray, text=gray, fill=white, minimum size=0.8cm] (data-1-\x) at (-6 + 4 * \x, -6) {$D_\a$};
					% Second data layer
					\foreach \a [count=\x] in {3, 4}
					  \node [draw=gray, text=gray, fill=white, minimum size=0.8cm] (data-2-\x) at (-9 + 2 * \x, -8) {$D_\a$};
					% First layer arrows
					\foreach \a in {1, 2}
					  \draw (node-1-1) -- (node-2-\a);
					% Second layer arrows
					\foreach \a in {1, 2}
					  \draw (node-2-1) -- (node-3-\a);
					\foreach \a in {3, 4}
					  \draw (node-2-2) -- (node-3-\a);
					% Third layer arrows
					\foreach \a in {1, 2}
					  \draw (node-3-1) -- (node-4-\a);
					\draw [draw=gray] (node-3-2) -- (data-1-1);
					\draw [draw=gray] (node-3-3) -- (data-1-2);
					\draw [draw=gray] (node-3-4) -- (data-1-3);
					\draw [draw=gray] (node-4-1) -- (data-2-1);
					\draw [draw=gray] (node-4-2) -- (data-2-2);
				\end{tikzpicture}
			\caption{A binary Merkle tree with five leaf nodes. While $H_i$ denotes the hash which is stored for the node with index $i$, $D_j$ denotes the data block with index $j$.}
			\label{fig:merkle-tree}
		\end{figure}

		\begin{figure}
			\centering
				\begin{tikzpicture}[line cap=round, line join=round, yscale=0.8]
					% First layer
					\node [circle, draw=black, fill=white, minimum size = 1cm] (node-1-1) at (0, 0) {$H_0^r$};
					% Second layer
					\node [circle, draw=black, fill=white, minimum size=1cm] (node-2-1) at (-4, -2) {$H_1^r$};
					\node [circle, draw=black, fill=white, minimum size=1cm] (node-2-2) at (4, -2) {$H_2^s$};
					% Third layer
					\node [circle, draw=black, fill=white, minimum size=1cm] (node-3-1) at (-6, -4) {$H_3^r$};
					\node [circle, draw=black, fill=white, minimum size=1cm] (node-3-2) at (-2, -4) {$H_4^s$};
					% Fourth layer
					\node [circle, draw=black, fill=white, minimum size=1cm] (node-4-1) at (-7, -6) {$H_7^s$};
					\node [circle, draw=black, fill=white, minimum size=1cm] (node-4-2) at (-5, -6) {$H_8^r$};
					% Second data layer
					\node [draw=gray, text=gray, fill=white, minimum size=0.8cm] (data-2-2) at (-5, -8) {$D_4^s$};
					% First layer arrows
					\foreach \a in {1, 2}
					  \draw (node-1-1) -- (node-2-\a);
					% Second layer arrows
					\foreach \a in {1, 2}
					  \draw (node-2-1) -- (node-3-\a);
					% Third layer arrows
					\foreach \a in {1, 2}
					  \draw (node-3-1) -- (node-4-\a);
					\draw [draw=gray] (node-4-2) -- (data-2-2);
				\end{tikzpicture}
			\caption{An exemplary Merkle tree integrity check. A superscript $s$ denotes that the respective value is given by the untrusted sender, whereas a superscript $r$ denotes a value which is computed by the receiver.}
			\label{fig:merkle-tree-proof}
		\end{figure}

		A \emph{Merkle} tree, or \emph{hash tree}, is a tree in which every leaf node stores the hash of a data block and every non-leaf node stores the hash of the concatenated hashes of its children \cite{merkle1987}.
		This structure enables efficient, partial integrity checks on large data sets.
			
		\autoref{fig:merkle-tree} shows a binary Merkle tree with five leaves.
		As the node with index $4$ is a leaf, the stored hash $H_4$ can be computed as
		\begin{equation}
			H_4 = h(D_0),
		\end{equation}
		where $h(\cdot)$ is the employed hash function and $D_0$ is the corresponding data block for this particular leaf node.
		In contrast, the node with index $3$ is a non-leaf node and its hash $H_3$ can hence be calculated as
		\begin{equation}
			H_3 = h(H_7 + H_8),
		\end{equation}
		where $H_7 + H_8$ denotes the concatenation of the hashes of nodes $7$ and $8$.
		Similarly, the hash of the root node $H_0$, also referred to as \emph{root hash} or \emph{Merkle root}, results as
		\begin{align}
			H_0 &= h(H_1 + H_2)\\
			&= h(h(H_3 + H_4) + h(H_5 + H_6))\\
			&= h(h(h(H_7 + H_8) + h(D_0)) + h(h(D_1) + h(D_2)))\\
			&= h(h(h(h(D_3) + h(D_4)) + h(D_0)) + h(h(D_1) + h(D_2))).\label{eq:merkle-recursive}
		\end{align}

		\autoref{eq:merkle-recursive} indicates that, due to the recursive nature of the hash computation, there is no possibility of altering any data block without changing the root hash as well.
		Therefore, it is sufficient to know the root hash in order to be able to verify the integrity of any data block in the tree.

		For instance, a sender wants to send data block $D_4$ to a receiver, which only knows the root hash $H_0$.
		Let $D_4^s$ denote the sent data block, where the superscript $s$ indicates that this data comes from the untrusted sender and could be different from the original $D_4$.
		In addition to the possibly altered data block $D_4^s$, the sender sends the hashes $H_2^s$, $H_4^s$ and $H_7^s$.
		With the given data block and the given hashes, the sender then resconstructs the corresponding Merkle tree to compute the root hash $H_0^r$ (see \autoref{fig:merkle-tree-proof}), where the superscript $r$ indicates that a hash is computed by the receiver:
		\begin{align}
			H_8^r &= h(D_4^s)\\
			H_3^r &= h(H_7^s + H_8^r)\\
			H_1^r &= h(H_3^r + H_4^s)\\
			H_0^r &= h(H_1^r + H_2^s).
		\end{align}
		If the computed root hash $H_0^r$ equals the known root hash $H_0$, the sender has successfully verified the integrity of the given data block ($D_4^s = D_4$) without knowing any other data block or non-root hash beforehand.
		This holds since, with the use of a cryptographic hash function (see \autoref{subsec:hashing}), the sender cannot determine a hash $H_7^\prime$ which, combined with an altered data block $D_4^\prime$, would yield the correct parent hash $H_3$.\footnote{For this, the sender would have to solve $H_3 = h(H_7^\prime + D_4^\prime)$ for $H_7^\prime$, which requires the infeasible inversion of the cryptographic hash function $h(\cdot)$.
		One might argue that the hash $H_3^r$ does not need to equal $H_3$ as long as the resulting root hash $H_0^r$ equals $H_0$ --- but this just propagates the problem of solving for an unknown input to $h(\cdot)$ to another level in the tree.}

