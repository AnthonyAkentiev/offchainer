\subsection{Client-side Application}
\label{subsec:approach-implementation-client}
\subsectionnames{Vincent Jonany}

\subsubsection*{Technologies}

We implemented our client side application using Node.js for the backend, and Express as our framework. We chose Node.js in our application because Node.js utilizes an event driven, non-blocking I/O model which helps with our integration with the Smart Contracts. Interactions with Blockchain technology, in this case Smart Contracts, result in long waiting time as transactions have to be mined. With the asynchronous property of Node.js, we can this time to asynchronously perform other functions in an effective manner. Moreover, Node.js allows us to easily use and integrate libraries to interact with Smart Contracts, such as Web3.js.

Web3.js contains specific functionality for the ethereum environment and enables you to interact with a local or remote ethereum node, utilizing a HTTP or IPC connection. These technologies have been proven to be the standard combination and have been used together in many communities. 

In addition to Web3.js, we also use and integrate Sequelize, a promised-based Object-Relational Mapping library that allows us easily create transactions with our database.

\subsubsection*{Hashing}

In the course of our implementation of the Merkle Tree, we have encountered many blockers, and one of bigger blockers is the implementation of the hash function in the client side application. Since the beginning, we have decided that we are going to have the same or similar function in our client side application as what the Smart Contract uses: SHA-3 hash function. We agreed on using SHA-3 (keccak), because it is the cheapest hash function available in a Solidity Smart Contract. 

Initially, we used the native default SHA-3 (keccak) hash function provided by the Web3.js library. However, we quickly encountered a major blocker. We first used simple strings to create a proof-of-concept. While the results of the hash function on both platforms, Javascript and Solidity, the representations are different: Solidity returns the hash result in a hex string representation ‘0x’, and follows 32 bytes of the hahs results, whereas Web3.js returns the result without ‘0x’. This gets more complicated as Merkle Tree requires hashing a hash. Solidity’s SHA-3 function automatically removes the first two characters ‘0x’ before hashing it, whereas in Web3.js we had to manually slice those characters out, and had to specify that the input has a hex encoding. 

The other huge blocker that we encountered includes the inconsistency of the data variables in Javascript and Solidity. Our next proof-of-concept includes incrementing a counter value in the Smart Contract, and that counter value is kept off-chained. Immediately, we received two different hash results from hashing the counter value, an integer, from Web3.js and Solidity. The problem is that when we pass in a number literal in Solidity’s SHA-3 function, it will convert the integer to the smallest possible integer type, in this case, a literal value of 1 will be considered as uint8. Hence, the solution requires us to keep the bytes representation of the input consistent. For example, we had to left-pad the byte representation of the integer by 2 hex characters, which calculates to 8 bytes, to match the representation of uint8 in Solidity. 

The solutions mentioned are not the best, as our implementation becomes very complex and dirty; we have to make left-pad 64 hex characters to the input to match the representation of uint256 variable in Solidity before hashing it using Web3.js. Fortunately, we found a third party library that provides SHA-3 (keccak) function that matches the characteristics of the function in Solidity. <https://web3js.readthedocs.io/en/1.0/web3-utils.html> It allows us to state the exact Solidity data type that we want to hash it as, and it also takes care of the appending and the pruning of the extra ‘0x’ hex characters of the hash results.

\subsubsection*{Revert}

The client side application is also responsible in reverting the previous Smart Contract transaction in case of error when updating results, such as new computed root hash from Smart Contract, into the database. In case of error, the client side application will create a new transaction for Smart Contract to revert it to the previous stored root hash.

\subsubsection*{Gas Cost Problem}

As mentioned in the Architecture section, we have taken the computation burden of creating the initial Merkle Tree in the client side application. This initial Merkle Tree creation process requires all of the data, the leaf nodes, to be present. Hence, we were blocked by the gas cost limit<link the problem> error when we were doing all of this inside the Smart Contact. The gas cost overhead does not happen during the creation of the tree, rather it happens when we send all the leaves as parameters in the Smart Contract transaction. 

We do try to keep as many responsibilities in the Smart Contracts as possible, because that will decrease the external trust and dependencies on the client side application. However, in the end, with our limited resources and the urgent need to continue on with our implementation and testing, we decided to move the initial creation of the Merkle Tree to the client side application.
