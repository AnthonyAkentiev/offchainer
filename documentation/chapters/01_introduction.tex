\chapter{Introduction}
\chapternames{Tarek Higazi, Dukagjin Ramosaj, Vincent Jonany}
One of the most significant technological achievements of this decade is the invention of blockchain technology. In the span of 10 years this technology has sprouted a new concept which the world has come to know as decentralized cryptocurrencies, and which today hold a combined value of hundreds of billions of US dollars \cite{relatedWork40}.

The blockchain architecture gives us a unique combination of benefits which include data integrity, transparency, security and peer-reviewed actions. 
“Blockchain-based applications, however, may also suffer from high computational and storage expenses, negatively impacting overall performance and scalability.” \cite{Eberhardt2017}

In recent years, blockchain technology has not only been utilized for cryptocurrencies, but rather to create a development platform as well. The concept of smart contracts was developed by Ethereum, a platform which, alongside its role in cryptocurrencies, is also known for the Ethereum Blockchain Development platform which utilizes blockchain. Through this concept, many great ideas and implementations have been produced. There is however a major issue which puts a limitation on the realization of these ideas. Just like any other program, there are local variables used to store different forms of data, yet the local variables in smart contracts are persistent on the blockchain network. This ultimately means that this data is replicated throughout all of the machines which participate in the blockchain network. Thus, storing data in smart contracts is expensive, and not practical.

The problem stated above calls for an approach which would enable users to store large amounts of data in a blockchain environment, while still benefiting from the features that come with it such as transparency, incorruptibility and data-integrity. The solution which we discuss and showcase in our paper is the approach of storing as much of the data as we can “off-chain” while preserving the integrity and the properties of the blockchain.

“Off-chaining” is basically the act of moving data or computation flows outside of the blockchain so that they can be stored or computed elsewhere \cite{Eberhardt2017}. However storing data in different databases, servers or any other location comes at a cost, and the blockchain’s core properties may not be possible to maintain. Ultimately, the framework should remain "trust-less", meaning that no external trust into a single actor in the system is required. 

In this paper we will describe how we developed a prototype which implements such an off-chaining approach with minimal impact to the blockchain’s properties. We came up with different use cases as examples of where it would be beneficial to use blockchain technology, and more importantly, where our off-chaining approach would make the most sense. We will also describe the implementation details together with the design decisions which were incorporated into the prototype, and explain how it works. In addition, we will lay out the results of our benchmarking and compare the costs and execution times of storing the data on-chain vs off-chain, and try to reach a conclusion as to where and when this solution would be applicable and useful. We will also provide insights on how one could develop this off-chaining approach even further.
