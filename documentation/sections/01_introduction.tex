\section{Introduction and Motivation}

One of the most significant technological achievements of this decade is the invention of blockchain technology. In the span of 10 years this technology has sprouted a new concept which the world has come to know as decentralized cryptocurrencies, and which today hold a combined value of hundreds of billions of US dollars \cite{relatedWork40}.

The blockchain architecture gives us a unique combination of benefits which combine data integrity, transparency, security and peer-reviewed actions. 
“blockchain-based applications, however, may also suffer from high computational and storage expenses, negatively impacting overall performance and scalability.” \cite{Eberhardt2017}

In the recent years, blockchain has not only been utilized for cryptocurrencies, but rather, it has also been used to create a development platform. The concept of smart contract is developed by Ethereum, a platform it not only well known for cryptocurrencies handling, but also for its aspects of Ethereum Blockchain Development which uses blockchain as a development background. Through this concept, there have been many great ideas and implementations produced. There is however a major issue, a blocker which limits the implementations of these ideas. Just like any other programs, there ought to be local variables to store forms of data, yet these local variables in smart contract is persistent and is saved in the blockchain network. This ultimately means that this data will be replicated throughout all machines that participate in the blockchain network. Thus, storing data in smart contracts is expensive, and not practical.

% Vincent
% Along with the rising popularity of cryptocurrencies, and the following soar in their prices, one of the main obstacles which has been seen is the rise in execution time. This obstacle is due to the ever increasing amount of data being processed and the increasing amount of transactions on the network, and this in turn fueled the rising values of the currencies even more.

% It is here where the problem we have focused on began to grow in significance. With the rising costs and execution times, the ability to store a lot of data on the blockchain becomes less practical.

The problem stated above calls for an approach that will enable users to store large amounts of data in a blockchain environment, while still benefiting from the features that come with it such as transparency, incorruptibility and data-integrity. The solution which we discuss and showcase in our paper is the approach of storing as much of the data as we can “off-chain” while preserving the integrity and the properties of the blockchain.

“Off-chaining” is basically the act of moving data or computation flows outside of the blockchain so that they can be stored or computed elsewhere \cite{Eberhardt2017}. However storing data in different databases, servers or any other location comes at a cost, and the blockchain’s core properties may not be possible to maintain. Ultimately, the framework should remain "trust-less", meaning that no external trust into a single actor in the system is required. Through this report we elaborate on our approach with minimal impact (or zero impact depending on the use case) on compromising the blockchain properties.

In this paper we will describe how we developed a prototype which implements the off-chaining approach with minimal cost to the blockchain’s properties. We came up with different use cases as examples of where it would make sense to use blockchain technology, and more importantly, where our off-chaining approach would make the most sense to be integrated in it. We will also show the implementation details together with the design decisions incorporated in the prototype, and how it works. In addition, we will lay out the results of our benchmarking and compare the costs and execution times of storing the data on-chain vs off-chain, and try to reach a conclusion on where and when this solution would be applicable and useful. We will also show the a dedicated section to which how one can go further with the off-chaining approach. 

\newpage
