\section{Introduction and Motivation}

One of the most significant technological achievements of this decade is the invention of Blockchain technology. In the span of 10 years this technology has sprouted a new concept which the world has come to know as decentralized cryptocurrencies, and which today hold a combined value of hundreds of billions of US dollars \cite{relatedWork40}. \\

The Blockchain architecture gives us a unique combination of benefits which combine data integrity, transparency, security and peer-reviewed actions. 
“Blockchain-based applications, however, may also suffer from high computational and storage expenses, negatively impacting overall performance and scalability.” \cite{Eberhardt2017}\\

Along with the rising popularity of crypto currencies, and the following soar in their prices, one of the main obstacles which has been seen is the rise in execution time, due to the ever increasing amount of data being processed and the increasing amount of transactions on the network, and this in turn fueled the rising values of the currencies even more.  \\

It is here where the problem we have focused on began to grow in significance. With the rising costs and execution times, the ability to store a lot of data on the Blockchain becomes less practical. This creates a need for coming up with new approaches and ideas for storing large amounts of data in a Blockchain environment, while still benefiting from the features that come with it such as transparency, incorruptibility and data-integrity. This would enable an approach which would free developers from being limited or constrained by the execution times and costs associated with storing, editing and updating such large amounts of data. The solution which we discuss and showcase in our paper is the approach of storing as much of the data as we can “off-chain” while preserving the integrity and the properties of the Blockchain.\\

“Off-chaining” is basically the act of moving data or computation flows outside of the Blockchain so that they can be stored or computed elsewhere \cite{Eberhardt2017}. However storing data in different databases, servers or any other location comes at a cost, and the Blockchain’s core properties may not be possible to maintain. Ultimately, the framework should remain "trust-less", meaning that no trust into a single actor in the system is required. Through this report we elaborate on our approach with minimal impact on compromising the blockchain properties.\\

In this paper we will describe how we developed a prototype which implements off-chaining with minimal cost to the Blockchain’s properties. We came up with different use cases as examples of where it would make sense to use Blockchain technology, and where the required amount of data would be too large to practically store on-chain.\\
We will show how our prototype was designed and how it functions. In addition, we will lay out the results of our benchmarking and compare the costs and execution times of storing the data on-chain vs off-chain, and try to reach a conclusion on where and when this solution would be applicable and useful. 


\newpage
