\section{Introduction and Motivation}

One of the most significant technological achievements of this decade is the invention of Blockchain technology. In the span of 10 years this technology has sprouted a new concept which the world has come to know as crypto currencies, and which today hold a combined value of hundreds of billions of US dollars. \\

The Blockchain architecture gives us a unique combination of benefits which combine data integrity, transparency, security and peer-reviewed actions.
“Blockchain-based applications, however, may also suffer from high computational and storage expenses, negatively impacting overall performance and scalability.” [ET 2017]\\

Along with the rising popularity of crypto currencies, and the following soar in their prices, one of the main obstacles is the rise in execution time, due to the ever increasing amount of data being processed and the increasing amount of traffic. This in turn fueled the rising values of the currencies even more.  \\

It is here where the problem we have focused on began to grow in significance. With the rising costs and execution times, the ability to store a lot of data on the Blockchain becomes less practical. \\

This creates a need for coming up with new approaches and ideas for storing large amounts of data in a Blockchain environment, while still benefiting from the features that come with it such as transparency, incorruptibility and data-integrity, this as to not be overwhelmed with the execution times and costs associated with such large data storage, and consequently the same problems when it comes to editing and updating this data.  \\


The solution which we discuss and showcase in our paper is the approach of storing as much of the data as we can “off-chain” while preserving the integrity and the properties of the Blockchain.\\

“Off-chaining” is the act of moving data or computation flows outside of the Blockchain so that they are stored or computed elsewhere. However storing data in different databases, servers or any other location comes at a cost, and the Blockchain’s core properties may not be possible to maintain. Ultimately, the framework should remain "trust-less", as in no unequivocal trust is required.\\

In this paper we will elaborate on how we developed a prototype which implements off-chaining to a Relational Database while compromising the Blockchain’s properties as little as possible. We came up with different use cases where it would be practical to make use of Blockchain technology, and where the amount of data which needed storing would be too large to practically store on-chain.\\
We will lay out the results of our benchmarking where we compared the costs and execution times of storing the data on-chain vs off-chain, and offer conclusions on where and when the presented off-chaining solution would be applicable and useful.


\newpage
