\subsection{Technologies} \label{sssec:technologies}


\subsubsection{Solidity Truffle and Ganache}


\subparagraph{Solidity}
Being as one of the most used contract-oriented high-level programming language for executing smart contracts. It was inspired by C++, Python, and JavaScript and is intended to focus on the Ethereum Virtual Machine (EVM). Solidity is compiled to bytecode that is executable on the EVM. Through Solidity, developers can write applications that enable self-enforcing business rationale exemplified in smart contracts, leaving an unchangeable and definitive record of transactions. Solidity is a statically typed, which inheritance is supported, libraries and complex user characterized types among other features \cite{relatedWork28}.

\subparagraph{Truffle}
The most popular development framework for Ethereum. It is written in JavaScript, therefore, being more flexible and providing numerous built-in features that assist and make the development easier. Truffle aims to handle the process of compilation, linking, deployment and binary management of smart contracts. Furthermore, it supports network management for deployment to private and public networks, its interactive console – which provides access to your built-in contracts, and availability of having automated contract testing through Mocha and Chai JavaScript test framework. In the scope of this project, we decided to start our implementation in Truffle framework. One of the main reason was due to the advantage of Truffle having a wider and active community support and availability of online resources, than any other Ethereum development frameworks. Moreover, the possibility of truffle integration with NodeJS became an important and crucial advantage for our implementation due to our team’s advanced skills on JavaScript programming \cite{relatedWork29}.

\subparagraph{Ganache}
It originates from the former known TestRPC,  which was a local (virtual) Ethereum blockchain for development purposes. It simulates the blockchain network, allowing you to make RPC calls in it without the need of actually mining the blocks. Ganache was developed by Truffle and is included in their latest development in Truffle Suite. The release of Ganache was positively anticipated across the whole Truffle framework, specifically in performance aspects - being much faster. Moreover, a huge advantage was providing an interface to the developer with the details of all the transactions being mined and their costs which further improved the user experience of the whole development platform itself \cite{relatedWork30}.


\subsubsection{Node.js, Express, Web3.js, Mocha and SuperTest}

\subparagraph{Node.js}
Being an open source cross-platform based on Chrome's JavaScript runtime for effortlessly constructing quick, scalable applications. Node.js utilizes an event driven, non-blocking I/O model which that makes it lightweight and effective, ideal for data-intensive real-time applications that keep running throughout distributed devices. Through event-driven property, meaning that the server reacts at the time when an event occurs therefore allowing us to create easily scalable, fast and real-time applications \cite{relatedWork31}

\subparagraph{Express}

Being a project from a node.js foundation, which is a quick, moderate web application framework for Node.js. It is intended for building web applications and is the true standard server system for Node.js. In the scope of our project we use Express as a basic routing layer built over the base of Node.js HTTP server that manages a server and the routes. It gives declarative routing without the need of doing “switch” or “if” statements or any additional functions, into a fundamental middleware design \cite{relatedWork32}.

\subparagraph{Web3.js}
An aggregation of libraries which contain specific functionality for the ethereum environment and enables you to interact with a local or remote ethereum node, utilizing a HTTP or IPC connection \cite{relatedWork33}.

\subparagraph{Mocha and SuperTest}

Mocha is a simple fast and extensible JavaScript test framework running on Node.js and in the browser. It is used for asynchronous testing, making unit and integration testing simple. Mocha runs tests serially, which allows reporting to be moderately flexible and accurate while matching uncaught exceptions to the correct test cases \cite{relatedWork34}. SuperTest is used for simple and fast testing of APIs. In the scope of our project, we used Mocha and SuperTest testing framework for our implementation of automated benchmarking.


\subsubsection{PostgreSQL and Sequelize}
\subparagraph{PostgreSQL}
An open source and highly scalable object-relational database system, which runs on all major operating systems including Linux, UNIX (AIX, BSD, HP-UX, macOS, Solaris), and Windows. It is completely ACID obedient, has full assistance for foreign keys, joins, views, triggers \cite{relatedWork35}.

\subparagraph{Sequelize}
A promise-based Node.js ORM for PostgreSQL, MySQL, SQLite and Microsoft SQL Server. It highlights strong transaction support, relations, read replication etc. \cite{relatedWork36}.


\subsubsection{Build-up/ Deployment Tools}
\subparagraph{Docker}
A deployment tool intended to make it less demanding to create, deploy, and run applications by using containers. Containers enable a developer to bundle up an application with all the parts it needs, such as libraries and different dependencies, and deliver everything out as one package. Thusly, because of the container, the developer can be certain that the application will run the same way on other machines, independently of their different operating system or any customized settings that the other machine may have. This gives a huge performance boost and decreases the extent and the size of the applications itself \cite{relatedWork37}.
