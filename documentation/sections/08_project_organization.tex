\section{Project Organization}

This chapter aims to provide an overview on our internal project organization. Firstly, we are going to cover our software development cycle. We will then provide brief descriptions on the methods and technologies we used to help us maintain the best practices in our software development cycle. 

We chose to follow an agile software development methodology and we used SCRUM as our framework. We mainly chose SCRUM due to the nature of our project - it is a highly complex and potentially huge project, and we needed constant guidance from our supervisor. SCRUM encourages short, iterative sprints, weekly meetings, and daily stand-ups. In addition to that, the number of members in the group seems very appropriate for SCRUM. Hence, we made SCRUM the backbone of our project organization. 

As we have been given a fixed schedule for the project, we were able to create two iterative cycles, each cycle ending with a presentation of our current findings and results to everyone who takes part in the project. The cycle consists of brainstorming, research, implementation, and a demo. 

In addition to our quarterly presentations, every two weeks we demo the current prototype to our product owners, which are the project supervisors. Every week the team members, excluding the previous roles mentioned previously, meet two times a week for a general discussion and sprint planning respectively. We chose to have shorter sprints (weekly), because we learned that it was very easy for us to lose track from the project goal. Hence, the weekly meeting will help us to stay on track. Moreover, we also have daily virtual stand-up meetings. This aims to further prevent one another from going out of the scope, and for everyone to be more aware of what each member is currently doing, or if they are blocked with the current task, and require immediate reinforcements. 

We use several technologies and mechanisms to aid us in following SCRUM with the best practices. One of the most important tool we use is ClickUp. ClickUp is a project management platform that helps us manage and track tasks, stories, epics, and sprints. During our sprint planning, our project manager would create all the necessary elements for the developers to start with their tasks. For example, creating a new sprint board, with all the members' tasks. The developers must then proceed with adding their own descriptions, user stories, and acceptance criteria into their own tasks. At the same time, the project manager can move and add tasks into and out from the backlog, depending on the resource allocation of each member for that sprint. 

As the sprint starts, ClickUp still plays a huge role during the development phase. Another tool that we strongly leverage on is GitHub, a version control for collaboration platform. Since each task is isolated, ClickUp allows users to track each commit and/or branch to each task, making our pull-requests organized, understandable and also isolated.

As a team, we also created a guideline on how to better isolate and organize branches, commits and ultimately the tasks themselves. For every task created on ClickUp, there is a unique identification which is used by us to link the name of the branches and commits. For example,

\begin{itemize}
\item For branches: \_\#\{task id\}\_short\_description\_of\_branch.
\item For commits: \#\{task id\}\_commits\_message.
\end{itemize}


Once developers are done with their features, they create a pull request and they can move that specific task to the "QA" category in ClickUp for another developer to review the pull request. We make sure that we have at least two approvals before the pull request can be merged, and only that assigned reviewer can move the task to "Done".

We heavily rely on Slacks and Google Hangouts for our day-to-day communication between team members, including the product owners, and our daily stand-up calls or messages. 

Lastly, we use Google Drive to store our presentations, research documents, schedules, and most importantly, the meeting protocols. These protocols are created whenever there are more than three people meet together for the project. The protocol should contain all the members present during that meeting, the agenda of the meeting, discussions, and next steps. The protocol helps project members to look back on what have been discussed in regards to design decisions and current of future tasks. 

